
\documentclass{article}
\usepackage{hyperref}
\usepackage{amsmath}
\usepackage{amssymb}
\usepackage{mathtools}
\usepackage{draftwatermark}


\SetWatermarkText{DRAFT}
\SetWatermarkScale{6}
\SetWatermarkLightness{0.95}

\title{A Consideration of Inflatable Circles}

\author{Robert L. Read
  \thanks{read.robert@gmail.com}
  email: \href{mailto:read.robert@gmail.com}{read.robert@gmail.com}\\
Megan Cadena
  \thanks{megancad@gmail.com}
  email: \href{mailto:megancad@gmail.com}{megancad@gmail.com}
  }

\begin{document}

\maketitle

\section{Introduction}
This is a study of the basic math of inflatable spheres as a tool for soft robotics.
We begin with a study in two dimension to simplify the problem.
Our final goal is to analyze three dimensional soft robots composed of inflatable
spheres.

\begin{figure}
     \centering
     \includegraphics[width=0.80\textwidth]{figures/FixedAxisCircles.png}
     \caption{Problem I: Fixed Circles Centers}
  \label{fig:fixed}
\end{figure}

\section{Problem I: Circles in Fixed Postions}

A very simplified version of the problem is to imagine that a two-dimensional
plane.  Instead of spheres, we will assume we have circles of changable radius.
This is in fact realistic of a soft robot constrained to a plane.




Eventually we hope to have circles pressing against each other, or tangent
or ``kissing''. However, the problem is a bit simpler if we assume we have
two circles, each of which is constrained to have its center on the a vertical
line (see Figure \ref{fig:fixed}.)  We place the circle $C_1$ with radius $r_1$ on the $x = -1$ line, and
assume that it rests on a shelf or plane on the $x$-axis.
Assume the $C_2$ circle whose radius if $r_2$ is on the $x = 1$ line.

Let $A$ the intersection of the tangent line supported
by the inflatable circles with the $x$-axis. Call the distance of $A$
on the $x$-axis $x$. Let $\psi$ be the angle formed by the
circle centers with the $x$ axis (measured counterclockwise).

\begin{align}
  \tan{\psi} &= \frac{r_1}{x-1} = \frac{r_2}{x+1}  \\
  r_2(x - 1) &= r_1(x+1) \\
  r_2x - r_2 &= r_1x + r_1 \\
  r_2x - r_1x &= r_1 + r_2 \\
  x(r_2 - r_1) &= r_1 + r_2 \\
  x &= \frac{r_1+r_2}{r_2 - r_1} \\
  \tan{2 \psi} = \frac{y}{x} \\
  2 \psi = \arctan{\frac{r_1}{\frac{r_1+r_2}{r_2-r_1} - 1}} \\
     \psi = \frac{\arctan{\frac{r_1}{\frac{r_1+r_2}{r_2-r_1} - 1}}}{2} \\
\end{align}





\section{Problem II: Tangent Circles}

\end{document}
