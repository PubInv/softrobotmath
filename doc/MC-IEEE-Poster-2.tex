%%
%% This is file `tikzposter-example.tex',
%% generated with the docstrip utility.
%%
%% The original source files were:
%%
%% tikzposter.dtx  (with options: `tikzposter-example.tex')
%%
%% This is a generated file.
%%
%% Copyright (C) 2014 by Pascal Richter, Elena Botoeva, Richard Barnard, and Dirk Surmann
%%
%% This file may be distributed and/or modified under the
%% conditions of the LaTeX Project Public License, either
%% version 2.0 of this license or (at your option) any later
%% version. The latest version of this license is in:
%%
%% http://www.latex-project.org/lppl.txt
%%
%% and version 2.0 or later is part of all distributions of
%% LaTeX version 2013/12/01 or later.
%%

 \documentclass[25pt, a0paper, portrait, margin=0mm, innermargin=15mm,
     blockverticalspace=15mm, colspace=15mm, subcolspace=8mm]{tikzposter} %Default values for poster format options.


\makeatletter
\def\title#1{\gdef\@title{\scalebox{\TP@titletextscale}{%
\begin{minipage}[t]{\linewidth}
\centering
#1
\par
\vspace{0.5em}
\end{minipage}%
}}}
\makeatother

\usepackage{hyperref}
\usepackage{amsmath}
\usepackage{amssymb}
\usepackage{mathtools}


 \tikzposterlatexaffectionproofon %shows small comment on how the poster was made at bottom of poster

 % Commands
 \newcommand{\bs}{\textbackslash}   % backslash
 \newcommand{\cmd}[1]{{\bf \color{red}#1}}   % highlights command


 % Title, Author, Institute



 \title{ Continued...}

 % -- PREDEFINED THEMES ---------------------- %
 % Choose LAYOUT:  Default, Basic, Rays, Simple, Envelope, Wave, Board, Autumn, Desert,
 \usetheme{Default}
\usecolorstyle[colorPalette=Default]{Default}


\begin{document}
\maketitle

     \begin{columns}
       \column{0.5}
     \block{Three Sphere Math}{

We seek to compute the normal of the tangent plane.
Observe that this plane is tangent to all three cones.
Observe that the $AB$ cone intersects
the $A$ sphere in a circle on the surface of the $A$ perpendicular to and centered on the $X$ axis.

A vector of length $r_A$ that begins pointing in the $X$ direction and is rotated counterclockwise
about the $Z$-axis by $\pi/2 - \alpha$.

However, we must rotate this vector about the $X$ axis by an unknown amount $\theta$ in order
to bring capture the tilt which is not purely a rotation about the $Z$ axis.
Since this angle is computed in the $ZY$ plane, we compute a projection of the point $V$ into
that plane, forming a triangle in the $ZY$ plane.  Call this point $I_z$, the intersection
of the apex line with the $Z$-axis.

\begin{align}
 \phi &= \arcsin{\frac{r_c}{I_z}}\\
\theta &= \frac{\pi}{2} - (\frac{\pi}{2} - \phi)\\
\theta &= \phi\\
I_z &= U_x(\frac{V_z}{(U_x - V_x})
\end{align}

     }
        \end{columns}
          \begin{columns}
       \column{0.5}
       \block{Future Work: Inflatable Stewart Platform}{
         By constructing soft mechanism that serves the same positioning perfomed
         by a mechanical Stewart Platform\cite{wiki:stewart}, we might build
         machines scalable up or down that that are gentle enough to be used
         for medical purposes.

         By solving the problem posed as an exercise in 1881\cite{payne1881} with
         closed-form expressions as shown here, we allow the possibility of
         computing the derivative of the tilt with respect to change of the radii.
         This allows control of the tilt by changing the pressure
         within the spheres.
    }
       \column{0.5}
       \block{Future Work: A Soft Tentacle}{
         By stacking such mechanisms, we propose to make a soft {\em tentacle}.
         By composing the derivative of tilt with respect to many such platforms,
         we may construct a Jacobian which allows positioning of the tentacle and
         even motion planning.

         Such a tentacle could be used as an endoscope or arthroscope.
         Because potentially scalable down to minute sizes,
         arterial catheterization may be possible.
    }
          \end{columns}

                 \begin{columns}
       \column{0.5}
          \block{Future Work: Joule Heating Phase Change for Inflation}{
            Although inflatable spheres could be controlled by pneumatics,
            it would be more elegant to build a sphere that inflates not
            by air tubes, but with a simple electrical connection, consisting
            of two wires.

            Although a gas changes pressure when heated, the change in
            pressure is proportional to the absolute temperature.
            Doubling this temperature is relatively impractical.
            However, water or alcohol can be vaporized at practically
            low temperatrues. We hope to design way to add simple heating wires
            inside an inflatable sphere in order to accomplish a phase change,
            and therefore a drastic pressure change,
            with a simple application of voltage.
          }
          \end{columns}

  \block{References}{
    \bibliographystyle{acm}
    \bibliography{softrobotmath}
  }



 \end{document}




\endinput
%%
%% End of file `tikzposter-example.tex'.
